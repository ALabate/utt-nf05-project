\chapter{Conclusion}
    \paragraph{}
        La réalisation de ce projet a eu à la fois un apport théorique, technique et humain sur les connaissances apportées par NF05. C'est donc une mise en situation réaliste nécessitant la mise en pratique de nos trois grands axes de connaissances universitaires et permettant de renforcer nos acquis.

    \paragraph{}
        Dans un premier temps, la délimitation du sujet (prise en compte des contraintes) et les objectifs à atteindre (création du cahier des charges) ainsi que les choix technologiques permettent d'avoir un vrai aperçu du travail d'ingénieur.
        \\La réalisation technique, en plus de renforcer les acquis de cours, a ouvert notre champ de connaissance en nous permettant de découvrir et d'utiliser d'autres technologies.
        \\Enfin, la réalisation d'un projet en groupe met en évidence les différents problèmes liés au travail collaboratif et nous a permis de mettre en oeuvre des solutions efficaces, et ainsi d'obtenir des compétences essentielles pour notre futur métier d'ingénieur et notre travail en entreprise.

    \paragraph{}
        En guise de conclusion sur le programme en lui même, nous pouvons dire que la plupart des objectifs qui étaient fixés ont été remplis. Le cahier des charges est entièrement rempli; nous regrettons néanmmoins de ne pas avoir pu implémenter une synthaxe plus ``naturelle'' quant à l'écriture des matrices. Une idée que nous avions eue était d'avoir une représentation graphique des matrices en tant réelle en utilisant de l'html/css dans un applet webkit. Par manque de temps, nous n'avons pas pu le faire. Le programme est également modulaire: nous voulions rajouter la gestion des nombres complexes, par exemple, et comme pour l'idée précédente, c'est le temps qui nous a fait défaut.
