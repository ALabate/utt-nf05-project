\chapter{Introduction}
    \paragraph{}
        Dans le cadre de l'UV NF05, ``Introduction au language C'', nous avons été amenés à mener à bien un projet de développement logiciel en language C (ou dérivé).
        Le logiciel en question, imposé, est un logiciel graphique de calcul matriciel. Il doit être capable de gérer des matrices de différentes tailles,
        les contenir sous formes de variables, les faire interagir au travers d'opérations mathématiques simples, et de pouvoir sauvegarder l'état courant du programme.
        Nous avons décidé de remplir également les objectifs supplémentaires, à savoir l'imbrication des opérations et la résolution d'équations simples.
    
    \paragraph{}
        Dans ce document, nous exposerons les différents outils et processus utilisés, ainsi que les difficultés rencontrées lors de la réalisation de ce
        projet.\\ 
        Ainsi, nous verrons dans un premier temps les choix technologiques que nous avons faits, puis les différents algorithmes utilisés pour résoudre
        les problèmes rencontrés. Enfin, nous présenterons le fonctionnement du programme.\\
        Le code est disponible, en tant qu'annexe, à la fin du document; il l'est également sur internet\footnote{https://github.com/ALabate/utt-nf05-project}: en effet, ce projet se veut libre. Il est ainsi librement accessible et modifiable\footnote{Sous license MIT}.